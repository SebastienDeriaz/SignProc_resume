\documentclass[resume]{subfiles}


\begin{document}
\section{Algèbre linéaire}
\subsection{Matrices}
$$x=\begin{bmatrix}
x(0)\\x(1)\\\vdots\\x(N-1)
\end{bmatrix}$$
$$A=\begin{bmatrix}
a_{11} & a_{12} & \cdots & a_{1m}\\
a_{21} & a_{22} & \cdots & a_{2m}\\
\vdots & \vdots & \ddots & \vdots\\
a_{n1} & a_{n2} & \cdots & a_{nm}
\end{bmatrix}$$
$$y(n)=h^Tx(n)=x^Th\qquad (\text{un élément à la fois})$$
\subsubsection{Transposée hermitienne}
$$\left(A^{H}\right)^{H}=A$$
$$(A+B)^{H}=A^H+B^H$$
$$(AB)^H=B^HA^H$$
\subsubsection{Rang}
Le rang d'une matrice $A$ donne le nombre de colonnes linéairement indépendantes qu'elle contient. Si le rang est \textbf{plein} alors
$$\text{rank}(A)=\rho(A)=n$$
\subsubsection{Inverse}
$$(AB)^{-1}=A^{-1}B^{-1}$$
$$(A^{H})^{-1}=(A^{-1})^{H}$$
Inversible si $\det(A)\neq 0$
\subsubsection{Déterminant}
$$\det(AB)=\det(A)\det(B)$$
$$\det(A^{T})=\det(A)$$
$$\det(\alpha A)=\alpha^{n}\det(A)$$
$$\det(A^{-1})=\frac{1}{\det(A)}$$
\subsubsection{Pseudo-inverse}
$$A\in \mathbb{R}_{n\times m}\qquad n<m$$
$$A^{+}=A^{H}(AA^{H})^{-1}$$
Permet de calculer la solution minimale
$$x=A^{+}b$$
\subsubsection{Norme}
$$\abs{\abs{x}}_1=\sum_{i=1}^{N}\abs{x_i}$$
$$\abs{\abs{x}}_2=\sqrt{\sum_{i=1}^{N}\abs{x_i}^2}$$
Permet aussi de calculer la distance : $\abs{\abs{x-y}}_2$
$$\abs{\abs{x}}_{\infty}=\max\abs{x_i}$$
\subsubsection{Produit scalaire}
$$\langle a,b \rangle = a\cdot b=\abs{\abs{a}}\abs{\abs{b}}\cos\theta$$
\subsubsection{Espace vectoriel}
$$\bold{v}=\sum_{i=1}^{N}\alpha_i\bold{v}_i$$
Si la base est orthogonale et si on utilise autant de coefficients qu'il y a de points dans le signal, alors la reconstruction est parfaite.
\subsubsection{Projection}
$$\hat{b}=A^{+}x_0=P_Ab$$
Par exemple
$$\begin{bmatrix}
2 & 1\\
1 & 2\\
1 & 2
\end{bmatrix}\begin{bmatrix}
x_1\\x_2
\end{bmatrix}=\begin{bmatrix}
1\\1\\1
\end{bmatrix}$$
$$A^{H}A=\begin{bmatrix}
6 & 5\\5 & 6
\end{bmatrix}$$
Solution des moindres carrés donnée par
$$x_0=(A^{H})^{-1}A^{H}b=\frac{1}{11}\begin{bmatrix}
6 & -5\\-5 & 6
\end{bmatrix}\begin{bmatrix}
2 & 1 & 1\\
1 & 2 & 1
\end{bmatrix}\begin{bmatrix}
1\\1\\1
\end{bmatrix}=\frac{4}{11}\begin{bmatrix}
1\\1
\end{bmatrix}$$
$$e=b-Ax_0=\frac{1}{11}\begin{bmatrix}
-1\\-1\\3
\end{bmatrix}$$
\subsubsection{Valeurs propres}
$$Av=\lambda v$$
Pour trouver les valeurs propres, on cherche les racines de
$$p(\lambda)=\det(A-\lambda I)=0$$
$$V=\begin{bmatrix}
v_1 & v_2 & \cdots & v_n
\end{bmatrix}\qquad \Lambda=\text{diag}\lbrace \lambda_1, \lambda_2, ...,\lambda_n\rbrace$$
$$AV=V\Lambda$$
\subsection{Approximation}
Autant de lignes que de points et autant de colonnes que de fonctions de base
$$A=\begin{bmatrix}
1 & x_{11} & \sin(2\pi x_{11}) & \cos(2\pi x_{12})\\
1 & x_{11} & \sin(2\pi x_{11}) & \cos(2\pi x_{12})\\
\vdots & \vdots & \vdots & \vdots\\
1 & x_{1n} & \sin(2\pi x_{1n}) & \cos(2\pi x_{1n})
\end{bmatrix}$$
Vecteur au sens des moindres carrés :
$$\Theta=(A^{T}A)^{-1}A^{T}b=A^{+}b$$
\end{document}